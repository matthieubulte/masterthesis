\section{Introduction}

With the democratization of data collection and analysis, the field of statistics is faced with new challenges stemming from the increased quantity and complexity of collected data. Graphical models have emerged in many scientific fields as a tool to describe variables of interest and their interactions. In biology, graphical models are used to study genes regulatory networks to better understand the development of diseases. In statistical mechanics, the Ising model \cite{ising1925beitrag} was introduced as a simplistic model of ferromagnetism to study interactions of particles on a 2 dimensional grid.

A \textit{graphical model} is a statistical model associating the joint distribution of a random vector $X = (X_1, \ldots, X_p)$ to a graph $\G$. The graph describes the dependence structures possible within the graphical model: nodes of $\G$ represent entries of the random vector $X$, and missing edges represent conditional independence constraints between the entries of $X$. The graphical view of a statistical model allows to study properties of the model by combining a graph theoretic analysis of the associated graph to probabilistic arguments.

A special type of graphical models that we study in this thesis are \textit{Gaussian graphical models}. In a Gaussian graphical model, the random vector $X$ follows a multivariate Normal distribution with covariance matrix $\Sigma$ and mean $\mu$. Since the interactions between the entries of $X$ are fully specified by the covariance matrix, a Gaussian graphical model is a multivariate Normal model in which the covariance matrix $\Sigma$ is constrained by the associated graph. Before being associated to graphs, multivariate Gaussian models with constraints on the covariance matrix have long been studied under the name of \textit{covariance selection models} as introduced by Dempster \cite{10.2307/2528966}.

When studying graphical models, one might naturally be interested in statistical questions related to the structure of the associated graph. Consider a graphical model $\c{P}$ associated to a graph $\G$ and parametrized by a vector $\theta \in \Theta \subset \R^d$, that is, $\c{P} = \eset{ P_\theta : \theta \in \Theta }$. If $\G_0$ is a subgraph constructed by removing edges from $\G$, we call $\c{P}_0 \subset \c{P}$ the \textit{submodel} of $\c{P}$ associated to $\G_0$ and we have $\c{P}_0 = \eset{ P_\theta : \theta \in \Theta_0 }$ with $\Theta_0 \subset \Theta$. In this thesis, we will be interested in testing statistical hypothesis of the form
\begin{equation*}
    H_0 : \theta \in \Theta_0 \ \ \t{vs.} \ \ H_1 : \theta \in \Theta \setminus \Theta_0.
\end{equation*}
That is, we are interested in knowing whether the true graphical model is associated to the sub-graph $\G_0$.

A standard approach for testing such a problem is the likelihood ratio test. The likelihood ratio test provides a generic approach that can be applied in a wide variety of models, based on the difference in maximum log-likelihood attainable in each model. Assuming the null hypothesis $H_0$ is true and under mild conditions, it can be shown that the likelihood ratio statistic converges to a $\chi^2_d$ distribution, where $d$ is the number of restrictions imposed by $H_0$. This generic result has led to a large adoption of this method in many statistical settings. However, it was recognized early-on that the $\chi^2_d$ result might not work well in finite sample and could be vastly improved by considering corrected versions of the likelihood ratio statistics. In an influential paper \cite{bartlett1937properties}, Bartlett shows that the likelihood ratio could be adjusted with a multiplicative factor to improve the accuracy of the $\chi^2$ approximation. Later, similar methods exploiting higher-order expansions of the characteristic function were developed under the term of \textit{higher-order statistics} to construct asymptotic approximations and adjustments to statistics with high-accuracy in small samples. Barndorff-Nielsen and Cox \cite{barndorff1989asymptotic, cox1994inference} offer a wide overview of the development and application of such methods.

In this thesis, we are intersted in studying applications of higher-order approximations for testing subgraph null hypotheses of Gaussian graphical models. Eriksen \cite{eriksen1996tests} shows that the $\chi^2$ approximation behaves very poorly in small samples and proposes an alternative test statistic based on a transformation of the likelihood ratio for which accurate higher-order approximations can be derived. We present the work of Eriksen and empirically show that it can be employed in a setting where the dimension of the problem grows with the sample size.

We structure the thesis as follows. In Chapter 2, we provide an introduction to higher-order approximation methods and present multi-dimensional convergence proofs based on the modern presentation of Kolassa \cite{kolassa2006series}. In Chapter 3, we bring our attention to Gaussian graphical models and present selected results concerning the existence of the maximum likelihood estimator based on Uhler \cite[Chapter 9]{maathuis2018handbook}. We then apply higher-order methods to submodel selection in Gaussian graphial models following the work of Eriksen \cite{eriksen1996tests}. Finally, we numerically study the behaviour of Eriksen's test statistic in a setting where the dimension of the problem is large compared to the sample size.

The source code for the approximations developed in this thesis are made available online\footnote{\url{https://github.com/matthieubulte/masterthesis/tree/master/code}}, alongside with the code to reproduce the figures and various experiments presented.