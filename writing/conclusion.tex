\section{Conclusion}

In this thesis, we studied an alternative statistical test to the likelihood ratio test for testing two alternative nested Gaussian graphical models. The new test introduced by Eriksen \cite{eriksen1996tests} involved exploiting the graph structure of the nested models to adapt general results from higher-order statistic.

We presented an introduction and analysis of the tools from higher-order statistics required to develop the new statistical test, resulting in the $p^*$ approximation to the distribution of the maximum likelihood estimator. We then studied the maximum likelihood problem in Gaussian graphical models and the conditions under which a solution exists. Studying both the primal and dual form of the likelihood maximization problem allowed us to link it to a matrix completion problem and exploit existence result from this area of linear algebra. We then derived the test presented in Eriksen \cite{eriksen1996tests} as a special case of the $p^*$ approximation in Gaussian graphical models.

A collection of simulations helped us compare the size and power of the new test presented and compare it to the likelihood ratio test. By studying different graph topologies and varying the dimensionality of the problem, we were able to numerically show that the new test, unlike the likelihood ratio test, is robust to an increase in the number of nuisance parameters. Furthermore, even when the number of parameters of interest grew, the new test was favorable compared to the likelihood ratio test in terms of size and power.

Results in higher-order statistics explain why the new test favorably compared to the likelihood ratio test in terms of size. However, these results under cover the behaviour of the test under the null hypothesis, and more work should be put in understanding the benefits of this test in terms of power. Furthermore, all proofs presented in this thesis only covered asymptotic theory in which the dimensionality of the problem is fix. Hence, it would be interesting to better understand why the new test appears to be robust to an increase of the number of nuisance parameters, and less sensitive than the likelihood ratio test to an increase of the number of parameters of interest. 