\section{Conclusion}

This thesis studied an alternative statistical test to the likelihood ratio test for testing two alternative nested Gaussian graphical models. The test introduced by Eriksen \cite{eriksen1996tests} involves exploiting the graph structure of the nested models to adapt general results from higher-order statistic.

We presented an introduction and analysis of the tools from higher-order statistics required to develop the new statistical test, resulting in the $p^*$ approximation to the distribution of the maximum likelihood estimator. We then studied the maximum likelihood problem in Gaussian graphical models and the conditions under which a solution exists. By studying both the primal and dual form of the likelihood maximization problem, we were able to link this problem to a matrix completion problem. This allowed us to exploit existence results from matrix completion theory in the context of likelihood maximization. Further, we derived the test presented in Eriksen \cite{eriksen1996tests} as a special case of the $p^*$ approximation in Gaussian graphical models.

A collection of simulations helped us understand the size and power of the Eriksen test and compare it to the likelihood ratio test. By studying different graph topologies and by varying the dimensionality of the problem, we were able to numerically show that the Eriksen test, unlike the likelihood ratio test, is robust to an increase in the number of nuisance parameters. Furthermore, the Eriksen test was favorable both in terms of size and power, even with increasing the number of parameters of interest, compared to the likelihood ratio test.

Results in higher-order statistics explain why the Eriksen test favorably compared to the likelihood ratio test in terms of size. However, these results do not cover the behaviour of the test under the null hypothesis. This leaves the opportunity to develop an understanding of the benefits of the Eriksen test in terms of power. Furthermore, all proofs presented in this thesis only cover asymptotic theory, in which the dimensionality of the problem is fix. It would be interesting to better understand why the test based on the $p^*$  approximation appears to be both robust to an increase of the number of nuisance parameters, and less sensitive than the likelihood ratio test to an increase of the number of parameters of interest. 