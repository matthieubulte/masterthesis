\begin{proof}
    Without loss of generality, we assume that, for $i \in [n]$, $X_i$ has mean 0 and a covariance matrix equal to the identity, see Remark \ref{rem-centering}. Let $\xi$ be the Fourier transform of $e_{k, n}(\cdot; \kappa(X))$, then by Corollary \ref{corr-fourier-inv}, we can bound, for any $y \in \Rp$, the absolute difference between $f_Y(y)$ and $e_{k, n}(y; \kappa(X))$ by
    \begin{equation} \label{eq-bound-abserr}
        |f_Y(y) - e_{k, n}(y; \kappa(X))| \leq (2\pi)^{-p}\intRp{t}{|\zeta(tn^{-1/2})^n - \xi(t)|},
    \end{equation}
    where $\zeta(tn^{-1/2})^n$ is the characteristic function of $Y$. Since both $\zeta$ and $\xi$ are $L^1(\Rp)$, the integral is well defined and provides a valid upper bound. We proceed by splitting the range of integration in two parts: one part in which $t$ is small such that Theorem \ref{thm-edge-inv-tech} can be applied, and the remaining part of the integral will be handled separately.

    By construction of the Edgeworth approximation, the Fourier transform of $e_{k, n}(\cdot; \kappa(X))$ corresponds to the function given in (\ref{eq-edgeworth-fourier}) of Theorem \ref{thm-edge-inv-tech}, where $k$ is replaced by $k+1$. Hence, for any $\varepsilon > 0$, there is $\delta > 0$ such that (\ref{eq-edge-bound}) holds and we can upper bound the \textit{small $t$} part of the right hand side of (\ref{eq-bound-abserr}) as follows,
    \begin{align*}
        \int_{B_2(\delta\sqrt{n})} &|\zeta(tn^{-1/2})^n - \xi(t)| \d t\\
        &\leq (2\pi)^{-p}\int_{B_2(\delta\sqrt{n})}\expf{-\frac{1}{4}\norm{t}_2^2}\left[ \frac{\epsilon\norm{t}_2^{k+1}}{n^{(k+1)/2-1}} + \frac{C_0^{k}\norm{t}_2^{3k}}{k!n^{(k+1)/2-1/2}} \right] \d t \\
        &\leq \frac{\epsilon C_1}{n^{k/2-1/2}}\expec{T}{\norm{T}^k_2} + \frac{C_2^{k}}{k!n^{k/2}}\expec{T}{\norm{T}_2^{3k}}
        = O(n^{(1-k)/2}),
    \end{align*}
    in which $T \sim N(0, 2\mathbb{1}_p)$ and $C_0, C_1, C_2 \in \R$ are constants that do not depend on $n$. 

    For the remaining part of the integral, where $\norm{t}_2 \geq \delta \sqrt{n}$, we bound the integral by the triangle inequality and consider each term separately,
    \begin{align*}
        \int_{\Rp \setminus B_2(\delta\sqrt{n})} &|\zeta(tn^{-1/2})^n - \xi(t)| \d t \\
        &\leq \int_{\Rp \setminus B_2(\delta\sqrt{n})} |\xi(t)| \d t + \int_{\Rp \setminus B_2(\delta\sqrt{n})} |\zeta(tn^{-1/2})^n| \d t\\
        &= I_1 + I_2.
    \end{align*}
    By construction of $\xi$, the integral $I_1$ has an exponential decay, thus $I_1$ is $O(n^{(1 - k)/2})$. As for the integral $I_2$, using that $|\zeta(t)| < 1$ for $t \neq 0$ and $|\zeta(t)| \rightarrow 0$ for $n \rightarrow \infty$ and any $t \in \Rp$, there exists $a \in (0, 1)$ such that for $n$ large enough, if $\norm{t}_2 \geq \delta\sqrt{n}$, then $|\zeta(tn^{-1/2})| \leq a$. Furthermore, by assumption of the existence of $f_Y$ and Lemma \ref{lem-char-integrable-convolution}, there exists $q > 1$ such that $\zeta^n \in L^q(\Rp)$. Thus,
    \begin{equation*}
        I_2
        \leq a^{n-q} \int_{\Rp \setminus B_2(\delta\sqrt{n})} |\zeta(tn^{-1/2})|^q \d t 
        \leq a^{n-q}\sqrt{n}\intRp{t}{|\zeta(t)|^q} 
        = O(\sqrt{n}a^n) = O(n^{(1-k)/2}),
    \end{equation*}
    which concludes the proof.
\end{proof}