\begin{proof}
    The idea of this proof is to rewrite the left hand side of the Equation (\ref{eq-edge-bound}) to be able to use Lemma \ref{lemma-series} and find suitable upper bounds on the remaining quantities. To that end, we define $u(t) = n u^*(t n^{-1/2})$ and $v(t) = nv^*(t n^{-1/2})$ where $u^*(t) = \log \zeta(t) + \frac{1}{2}\norm{t}_2^2$ and 
    \begin{equation*}
        v^*(t) = \sum_{m=3}^k \sum_{s \in S_p(m)} \frac{i^m\kappa_s t^s}{m!}.
    \end{equation*}
    We can then rewrite
    \begin{equation*}
        \zeta(t n^{-1/2})^n = \expf{n\left[\log \zeta(t n^{-1/2})\right]} = \expf{-\frac{1}{2}\norm{t}_2^2}\exp u(t)
    \end{equation*}
    and
    \begin{equation*}
        \xi(t) = \expf{-\frac{1}{2}\norm{t}_2^2}\sum_{l=0}^{k-2} \frac{v(t)^l}{l!}.
    \end{equation*}
    By Lemma \ref{lemma-series}, we can bound the left hand side of Equation (\ref{eq-edge-bound}) and have
    \begin{align*}
        &\abs{\zeta(t n^{-1/2})^n - \xi(t)} 
        = \expf{-\frac{1}{2}\norm{t}_2^2}\abs{\exp u(t) - \sum_{l=0}^{k-2} \frac{v(t)^l}{l!}}\\
        &\leq \expf{-\frac{1}{2}\norm{t}_2^2} \max \left\{ \exp{|u(t)|}, \exp{|v(t)|} \right\}\left(\abs{u(t) - v(t)} + \frac{|v(t)|^{k-1}}{(k-1)!}\right)
    \end{align*}
    We now continue to find suitable bounds on the resulting quantities. First note that both $u$ and $v$ have continuous derivatives in 0 of order up to $k$. Starting with $u^*$, let $s \in S_p(m)$ for $3 \leq m < k$, then
    \begin{align*}
        D^s u^*(0)
        &= \ddx{t_{s_1}}\ldots\ddx{t_{s_m}} \log \zeta(t) + \frac{1}{2}\norm{t}_2^2\bigg|_{t=0}\\
        &=  \ddx{t_{s_1}}\ldots\ddx{t_{s_m}} \log \zeta(t)\bigg|_{t=0}
            +
            \ddx{t_{s_1}}\ldots\ddx{t_{s_m}} \frac{1}{2}\norm{t}_2^2\bigg|_{t=0}
        \\
        &= (-i)^{-m}\kappa_s = i^m \kappa_s ,
    \end{align*}
    where the derivative of the 2-norm of $t$ is 0 because $|s| = m \geq 3$. Furthermore, we can compute the same derivatives of $v^*$,
    \begin{align*}
        \ddx{t_{s_1}}\ldots\ddx{t_{s_m}} v^*(t) \bigg|_{t=0}
        &= \ddx{t_{s_1}}\ldots\ddx{t_{s_m}} \sum_{m'=3}^k \sum_{s' \in S_p(m')} \frac{i^{m'}\kappa^{s'}t_{s'_1}\ldots t_{s'_{m'}}}{m'!} \bigg|_{t=0} \\
        &= \sum_{m'=3}^k \sum_{s' \in S_p(m')} \frac{i^{m'}\kappa^{s'}}{m'!} \ddx{t_{s_1}}\ldots\ddx{t_{s_m}}t_{s'_1}\ldots t_{s'_{m'}} \bigg|_{t=0}.
    \end{align*}
    The term $\ddx{t_{s_1}}\ldots\ddx{t_{s_m}}t_{s'_1}\ldots t_{s'_{m'}} |_{t=0} = 1$ if and only if $s'$ is a permutation of $s$, otherwise 0. Hence
    \begin{equation*}
        \ddx{t_{s_1}}\ldots\ddx{t_{s_m}} v^*(t) \bigg|_{t=0} 
        = \sum_{s' \in S_p(m)} \frac{i^m\kappa^{s'}}{m!}
        = i^m\kappa_s,
    \end{equation*}
    where the last equality holds since $\kappa_s = \kappa^{s'}$ for all permutation $s'$ of $s$. 
    This shows that all derivatives of order up to $k$ of $u^* - v^*$ (and hence also $u - v$) exist in 0 and are all equal to 0. Therefore, there exists a $\delta > 0$ such that for all $t \in \Rp$ with $\norm{t}_2 \leq \delta$
    \begin{equation*}
        \abs{u^*(t) - v^*(t)} \leq \epsilon\norm{t}_2^k
    \end{equation*}
    and if $\norm{t} \leq \delta \sqrt{n}$, this bound yields
    \begin{equation*}
        \abs{u(t) - v(t)} = n\abs{u^*(t n^{-1/2}) - v(t n^{-1/2})} \leq n\epsilon\norm{n^{-1/2}t}_2^k = n^{1-k/2}\norm{t}_2^k\epsilon.
    \end{equation*}
    Furthermore, choose $\delta$ small enough such that $\abs{u^*(t)} < \norm{t}_2^2/4$ for $\norm{t}_2 \leq \delta$. Then for $\norm{t}_2 \leq \delta\sqrt{n}$ we have
    \begin{equation*}
        \abs{u(t)} = n\abs{u^*(t n^{-1/2})} \leq n \norm{t n^{-1/2}}_2^2/4 = \norm{t}_2^2/4.
    \end{equation*}
    We observe that all derivatives of $v^*$ in 0 of first and second order are equal to 0 and that the third derivatives of $v^*$ are bounded. By Taylor's theorem, this allows us to find the following bound for all $\norm{t}_2 \leq \delta \sqrt{n}$
    \begin{equation*}
        |v(t)| = |nv^*(t n^{-1/2})| < C_p n \norm{t n^{-1/2}}_2^3 = C_p \norm{t}_2^3 n^{-1/2}
    \end{equation*}
    where
    \begin{equation*}
        C_p = \sup_{\substack{\norm{t}_2 \leq \delta, \\ s \in S_p(3)}} p^3\abs{ D^sv^*(t) }.
    \end{equation*}
    Hence, for a $\delta$ small enough and $t \leq \delta\sqrt{n}$ we have
    \begin{align*}
        &\abs{\zeta(t n^{-1/2})^n - \xi(t)} \\
        &\leq \expf{-\frac{1}{2}\norm{t}_2^2} \max \left\{ \exp{|u(t)|}, \exp{|v(t)|} \right\}\left(\abs{u(t) - v(t)} + \frac{|v(t)|^{k-1}}{(k-1)!}\right)\\
        &\leq \expf{-\frac{1}{2}\norm{t}_2^2}
            \max \left\{ 
                \expf{\frac{1}{4}\norm{t}_2^2},
                \expf{C \norm{t}_2^3 n^{-1/2}}
            \right\}\\
        & \times \left(
                \frac{\epsilon\norm{t}_2^k}{n^{k/2-1}}
                +
                \frac{C_p^{k-1} \norm{t}_2^{3(k-1)} }{n^{k/2-1/2} k!}
              \right)\\
        & \leq \expf{-\frac{1}{4}\norm{t}_2^2}\left[ \frac{\epsilon\norm{t}_2^k}{n^{k/2-1}} + \frac{C_p^{k-1}\norm{t}_2^{3(k-1)}}{(k-1)!n^{k/2-1/2}} \right].
    \end{align*}
\end{proof}