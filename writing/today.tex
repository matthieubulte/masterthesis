
% In Eriksen \cite{eriksen1996tests}, we work with the likelihood ratio statistic, denoted $Q$
% \begin{equation*}
%     Q = \frac{L(\hat\Omega_{\mathcal{G}_0)}}{L(\hat\Omega_{\mathcal{G}})} = \frac{|\hat\Omega_{\mathcal{G}_0}|^{n/2}}{|\hat\Omega_{\mathcal{G}}|^{n/2}}.
% \end{equation*}

% Standard MLE theory shows that the log-likelihood ratio $W = -2 \log{Q}$ is asymptotically $\chi^2_1$. Eriksen shows that a better statistic to look at is $Q^{2/n}$ which can be  shown to be asymptotically $\mathcal{B}(\frac{n - f(a) - 1}{2}, \frac{1}{2})$.

% Let's define the following functions
% \begin{align*}
%     f(q) &= 1 - P\left(\chi^2_1 \leq W(q)\right)\\
%     g_n(q) &= P\left(\mathcal{B}(\frac{n - f(a) - 1}{2}, \frac{1}{2}) \leq q^{2/n}\right)
% \end{align*}

% We observe numerically that $g_n$ converges to $f$ as $n \rightarrow \infty$. This means that the approximations resulting from the $\chi^2_1$ approximation to the distribution of $W$ and the $\mathcal{B}(\frac{n - f(a) - 1}{2}, \frac{1}{2})$ approximation to the distribution of $Q^{n/2}$ converge to eah other, independently of $p$. This means that the $Q^{n/2}$ approximation is only interesting for small $n$.

every experiment should be linked to a file and a commit number

things to try:
    - look at the beta approximation in the dense vs dense with one edge less, because in this cause f(a) ~ p
    - look at the beta approximation in the dense vs sparse testing == removing multiple edges
    - non-linear models ?





Introduction
    + Short intro to GGMs, the motivation
        - What are GGMs
        - Concrete example
        - Covariance selection model
    
    + Moderate dimensional setting
        - What is moderate dimension
        - Vhallenges in settings of moderate dimension
        - Example with failure of chi sq approximation

    + Teaser of higher-order statistics
        - Motivation and history of the topic
        - Mention some of the asymptotical results that are promised

Part 1. Higher-order statistics
    + Chapter 1. Short intro and motivation
        - Talk quickly about first order method, which is familiar to reader
        - Introduce a classical approximation example, not related to GGM
        - Use Bartlett adjustment, or something similar to demonstrate better approximation

    + Chapter 2. Basis for higher-order approximations (based on Kolassa 2006)
        - Notation, good intro to the statistical setup
        - Characteristic functions and inversion of characteristic function
        - General approximation theorem from expansions of the characteristic function 
        - Edgeworth expansion
            With examples, show flaws of absolute error coming from Edgeworth expansion
            Julia symbolic implementation can be nice to show there

    + Chapter 3. The Barnorff-Nielsen formula in exponential families
        - Tilted approximation in exponential families (Barndorff-Nielsen and Cox 1989)
            With example, show advantage of relative error
        - p* formula
            In this case it's really just mentioning it since the p* formula is the tilted approximation for exponential families by actually computing it
        - Some simple examples of higher-order statistics that have accurate approximation results
            Bartlett correction on simple examples, Barndorff-Nielsen and Cox 1994 has many good examples
    
Part 2. Gaussian Graphical Models
    + Chapter 4. Intro to graphical models
        - Directed vs undirected
        - Markov property
        - Decomposition of density

    + Chapter 5. Gaussian Graphical models
        - Introduce the model
        - Relation between precision matrix and independence
        - Conditions for existence of MLE (based on Handbook of Graphical Models, Chapter 9)
        - Computing the MLE of the precision matrix for a given graph (iterative proportional scaling / Julia implementation)

    + Chapter 6: Higher-order techniques for GGMs
        - Presentation of work in Erisken 1996
        - Presentation of numerical experiments on different topologies from the last months

Conclusion

Appendix A. Some notions on exponential families



