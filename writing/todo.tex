\section*{To do}

\begin{itemize}

\item {\color{red} Introduction
\begin{itemize}
    \item{+ Short intro to GGMs, the motivation
    \begin{itemize}
        \item- What are GGMs
        \item- Concrete example
        \item- Covariance selection model
    \end{itemize}}
    \item{+ Moderate dimensional setting
    \begin{itemize}
        \item- What is moderate dimension
        \item- Vhallenges in settings of moderate dimension
        \item- Example with failure of chi sq approximation
    \end{itemize}}
    
    \item{+ Teaser of higher-order statistics
    \begin{itemize}
        \item- Motivation and history of the topic
        \item- Mention some of the asymptotical results that are promised
    \end{itemize}}
\end{itemize}
}
\item{Part 1. Higher-order statistics
\begin{itemize}
    \item{\color{red} + Chapter 1. Short intro and motivation
    \begin{itemize}
        \item- Talk quickly about first order method, which is familiar to reader
        \item- Introduce a classical approximation example, not related to GGM
        \item- Use Bartlett adjustment, or something similar to demonstrate better approximation
    \end{itemize}}
    \item{\color{green} + Chapter 2. Basis for higher-order approximations (based on Kolassa 2006)
        \begin{itemize}
            \item ok - Notation, good intro to the statistical setup
            \item ok - Characteristic functions and inversion of characteristic function
            \item ok - General approximation theorem from expansions of the characteristic function 
            \item {ok - Edgeworth expansion
            With examples, show flaws of absolute error coming from Edgeworth expansion
            Julia symbolic implementation can be nice to show there}
        \end{itemize}}
        \item{\color{green} + Chapter 3. The Barnorff-Nielsen formula in exponential families
            \begin{itemize}
                \item {ok - Tilted approximation in exponential families (Barndorff-Nielsen and Cox 1989)
            With example, show advantage of relative error}
            \item {ok - p* formula
            In this case it's really just mentioning it since the p* formula is the tilted approximation for exponential families by actually computing it}
            \item {- Some simple examples of higher-order statistics that have accurate approximation results
            Bartlett correction on simple examples, Barndorff-Nielsen and Cox 1994 has many good examples}
        \end{itemize}}
\end{itemize}
}
\item{Part 2. Gaussian Graphical Models
    \begin{itemize}
        \item{\color{red} + Chapter 4. Intro to graphical models
    \begin{itemize}
        \item- Directed vs undirected
        \item- Markov property
        \item- Decomposition of density
    \end{itemize}}
    \item{\color{green} + Chapter 5. Gaussian Graphical models
    \begin{itemize}
        \item ok - Introduce the model
        \item ok - Relation between precision matrix and independence
        \item ok - Conditions for existence of MLE (based on Handbook of Graphical Models, Chapter 9)
        \item ok - Computing the MLE of the precision matrix for a given graph (iterative proportional scaling / Julia implementation)
    \end{itemize}}
    \item{\color{orange} + Chapter 6: Higher-order techniques for GGMs
    \begin{itemize}
        \item - Presentation of work in Erisken 1996
        \item - Presentation of numerical experiments on different topologies from the last months
    \end{itemize}}
\end{itemize}
}
\item \color{red} Conclusion
\end{itemize}
\newpage