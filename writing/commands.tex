
\DeclarePairedDelimiter\ceil{\lceil}{\rceil}
\DeclarePairedDelimiter\floor{\lfloor}{\rfloor}


%%%%%%%%%%%%%%%%%%%%%%%%%%%%%%%%%%%%%%%%%%%%%%%%%%%%%%%%%%%%%%%%%%%%%%%%%%%%
% Sugar
%%%%%%%%%%%%%%%%%%%%%%%%%%%%%%%%%%%%%%%%%%%%%%%%%%%%%%%%%%%%%%%%%%%%%%%%%%%%

\newcommand{\independent}{\perp\mkern-9.5mu\perp}
\newcommand{\notindependent}{\centernot{\independent}}
\newcommand{\norm}[1]{\left\lVert#1\right\rVert}
\newcommand{\logit}[1]{\text{logit}\left(#1\right)}
\newcommand{\ilogit}[1]{\text{logit}^{-1}\left(#1\right)}
\newcommand{\expec}[2]{\mathbb{E}_{#1}\left[#2\right]}
\renewcommand{\P}{\mathbb{P}}
\renewcommand{\d}{\text{d}}
\newcommand{\simiid}{\stackrel{iid}{\sim}}
\newcommand{\trarrow}[1]{\xrightarrow{\text{#1}}}
\newcommand{\R}{\mathbb{R}}
\newcommand{\C}{\mathbb{C}}
\newcommand{\E}{\mathbb{E}}
\newcommand{\V}{\mathbb{V}}
\newcommand{\N}{\mathbb{N}}
\renewcommand{\O}{\mathcal{O}}
\newcommand{\Sp}{{\mathcal{S}_p}}
\renewcommand{\S}{{\mathcal{S}}}
\newcommand{\G}{{\mathcal{G}}}
\newcommand{\ddx}[1]{\frac{\d}{\d #1}}
\newcommand{\Rp}{{\R^p}}
\newcommand{\F}{\mathcal{F}}
\renewcommand{\Finv}{{\mathcal{F}^{-1}}}
\newcommand{\abs}[1]{\left|#1\right|}
\newcommand{\intRp}[2]{\int_{\Rp} #2 \d #1}
\newcommand{\expf}[1]{\exp\left(#1\right)}
\newcommand{\expfc}[1]{\exp\left\{#1\right\}}
\newcommand{\logf}[1]{\log\left(#1\right)}
\newcommand{\sidenote}[1]{\footnote{\color{red}#1}}
\newcommand{\tr}{{\textrm{tr}}}
\newcommand{\trB}[1]{{\textrm{tr}\left[#1\right]}}
\newcommand{\eset}[1]{{\left\{#1\right\}}}
\newcommand{\cov}{{\textrm{Cov}}}
\renewcommand{\t}[1]{{\textrm{#1}}}
\renewcommand{\c}[1]{{\mathcal{#1}}}
%%%%%%%%%%%%%%%%%%%%%%%%%%%%%%%%%%%%%%%%%%%%%%%%%%%%%%%%%%%%%%%%%%%%%%%%%%%%
% Theorems setup
%%%%%%%%%%%%%%%%%%%%%%%%%%%%%%%%%%%%%%%%%%%%%%%%%%%%%%%%%%%%%%%%%%%%%%%%%%%%
\newtheoremstyle{def}
  {12pt}   % ABOVESPACE
  {6pt}   % BELOWSPACE
  {\normalfont}  % BODYFONT
  {0pt}       % INDENT (empty value is the same as 0pt)
  {\bfseries} % HEADFONT
  {.}         % HEADPUNCT
  {5pt plus 1pt minus 1pt} % HEADSPACE
  {}          % CUSTOM-HEAD-SPEC

\theoremstyle{def}
\newtheorem{theorem}{Theorem}[section]
\theoremstyle{def}
\newtheorem{lemma}[theorem]{Lemma}
\theoremstyle{def}
\newtheorem{corollary}[theorem]{Corollary}
\theoremstyle{def}
\newtheorem{example}[theorem]{Example}
\theoremstyle{def}
\newtheorem{assumption}[theorem]{Assumption}
\theoremstyle{def}
\newtheorem{remark}[theorem]{Remark}
\theoremstyle{def}
\newtheorem{definition}[theorem]{Definition}
\theoremstyle{def}
\newtheorem{proposition}[theorem]{Proposition}

\graphicspath{{figures/}}

\global\newcommand{\note}[1]{{\color{red}(#1)}}

\def\centerarc[#1](#2)(#3:#4:#5)% Syntax: [draw options] (center) (initial angle:final angle:radius)
    { \draw[#1] ($(#2)+({#5*cos(#3)},{#5*sin(#3)})$) arc (#3:#4:#5); }



%%
%% Julia definition (c) 2014 Jubobs
%%
\lstdefinelanguage{Julia}%
  {morekeywords={abstract,break,case,catch,const,continue,do,else,elseif,%
      end,export,false,for,function,immutable,import,importall,if,in,%
      macro,module,otherwise,quote,return,switch,true,try,type,typealias,%
      using,while},%
   sensitive=true,%
   alsoother={$},%
   morecomment=[l]\#,%
   morecomment=[n]{\#=}{=\#},%
   morestring=[s]{"}{"},%
   morestring=[m]{'}{'},%
}[keywords,comments,strings]%

\lstset{%
    language         = Julia,
    basicstyle       = \ttfamily,
    keywordstyle     = \bfseries\color{blue},
    stringstyle      = \color{magenta},
    commentstyle     = \color{ForestGreen},
    showstringspaces = false,
}


