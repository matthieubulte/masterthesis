\subsection{Preliminaries} \label{sec-prelim}

We begin with a brief introduction to elementary concepts in graph theory. Let $\G = (\Gamma, E)$ be a graph with nodes $\Gamma$ and edges $E$. In the sequel, we only consider unconnected and loopless graphs. For notational simplicity, we assume that the nodes $\Gamma$ are numbered. For $p = |\Gamma|$, we write $\G = ([p], E)$ and have $E \subset \eset{ \eset{i, j} : i, j \in [p], i \neq j }$.  We denote by $\t{bd}(i)$ the set of \textit{neighbours} of $i \in [p]$, that is $\t{bd}(i) = \eset{ j \in [p] : \eset{i, j} \in E \t{ and } j \neq i}$.  Note that it will be useful to treat the set of edges as a set of indices on a matrix. For this case, we introduce the \textit{augmented edge set}, in order to include indices refering to the diagonal entries of a matrix. The augmented edge set $E^*$ of $E$ is the set $E^* = E \cup \eset{ \eset{i} : i \in \Gamma }$, constructed by adding all possible loops in $\G$.

A graph $\G$ is said to be \textit{complete} if all pairs of distinct nodes are connected by an edge. A \textit{clique} of $\G$ is a set of nodes $C \subset [p]$ such that the subgraph $\G_C = (C, E_C)$ with $E_C = \eset{ \eset{i, j} \in E : \eset{i, j} \subset C }$ is complete. We denote by $\c{C}(\G)$ the set of cliques in $\G$.

An important class of graphs are \textit{chordal graphs}. A chordal graph $\G$ is a graph in which each cycle of length at least 4 has a \textit{chord}, which is an edge that is not part of the cycle connecting two nodes in the cycle.

Let $\G = ([p], E)$ and $M \in \R^{p \times p}$. We introduce the following notation.
\begin{itemize}
    \item If $e = \eset{i, j} \in E$, then $M_e = M_{ij}$.
    \item If $A, B \subset [p]$, then $M_{A, B}$ is the $|A| \times |B|$ matrix constructed by keeping rows labeled by the entries in $A$ and columns labeled by the entries in $B$.
    \item If $A \subset [p]$, then $M_A = M_{A, A}$.
    \item {
        If $A, B \subset [p]$, then $[M_{A, B}]^{[p]}$ is the $p \times p$ matrix with entries given by
        \begin{equation*}
            [M_{A, B}]^{\Gamma}_{ij} = \begin{cases}
                M_{ij} \ \t{ if } \eset{i, j} \in A \times B, \\
                0 \ \ \ \ \ \t{otherwise}.
            \end{cases}
        \end{equation*}
    }
\end{itemize}
Since most matrices manipulated in the sequel reference quantities related to nodes in a graph, we index matrices with respect to the nodes of the graph instead of row or column number of the matrix. For instance, if $A, B \subset [p]$ and $M \in \R^{p \times p}$, then following the notation introduced above, we have that
\begin{equation*}
    (M_{A, B})_{ab} = M_{ab} \ \ \t{for all } a \in A, b \in B.
\end{equation*}

We denote by $\S^p$ the set of $p \times p$ symmetric matrices and $\S^p_{\succ 0}$ the set of $p \times p$ positive definite symmetric matrices. Let $\G = ([p], E)$, we define the set of matrices $\S(\G) = \eset{ M \in \S^p : M_e = 0 \t{ if } e \notin E^* }$ and $\S_{\succ 0}(\G) = \eset{ M \in \S(\G) : M \t{ is positive definite } }$ or equivalently $\S_{\succ 0}(\G) = \S(\G) \cap \S^p_{\succ 0}$.