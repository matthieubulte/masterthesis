\subsection{The characteristic function and related quantities} \label{sec-charfun}

The \textit{characteristic function} is a central tool in studying probability distributions. In this section, we review general results about the characteristic function of a multivariate continuous distribution. Let $X$ be a random vector in $\Rp$. The characteristic function of $X$ is the function $\zeta : \Rp \rightarrow \C$ given by
\begin{equation*}
    \zeta(t) = \expec{}{\expf{it^\top X}}.
\end{equation*}
The characteristic function is an essential tool in studying distributions. Indeed, the following multidimensional extebnsion of \cite[Theorem 2.4.2]{kolassa2006series} shows that under regularity conditions on the characteristic function, the density of a random vector exists and can be expressed in terms of the characteristic function.

\begin{theorem} \label{thm-char-inversion}
    Let $X \sim P$ be a random vector in $\Rp$ with characteristic function $\zeta \in L^1(\Rp)$. Then, the density of $X$ exists and is given by
    \begin{equation} \label{eq-density-via-charfun}
        f(x) = \left(2\pi\right)^{-p} \intRp{t}{ \expf{-it^\top x}\zeta(t) }.
    \end{equation}
\end{theorem}

\begin{proof}
    Let $A \subset \Rp$ be a bounded rectangle $A = [a_1, b_1] \times \ldots \times [a_p, b_p]$ with $P(X \in \partial A) = 0$. By Theorem 3.10.4 in \cite{durrett_2019}, we have that
    \begin{equation*}
        P(X \in A) = \lim_{T \rightarrow \infty} \left(2\pi\right)^{-p}\int_{\left[-T, T\right]^p} \zeta(t) \prod_{k=1}^p \frac{\expf{-it_k a_k} - \expf{-it_k b_k}}{i t_k} \d t.
    \end{equation*}
    By rewriting various terms under the integral, one obtains
    \begin{align*}
        P(X \in A) 
        &= \lim_{T \rightarrow \infty} \left(2\pi\right)^{-p}\int_{\left[-T, T\right]^p} \zeta(t) \prod_{k=1}^p \frac{\expf{-it_k a_k} - \expf{-it_k b_k}}{i t_k} \d t \\
        &= \lim_{T \rightarrow \infty} \left(2\pi\right)^{-p}\int_{\left[-T, T\right]^p} \zeta(t) \prod_{k=1}^p \int_{a_k}^{b_k}\expf{-i t_k x_k} \d x_k \d t \\
        &= \lim_{T \rightarrow \infty} \left(2\pi\right)^{-p}\int_{\left[-T, T\right]^p} \zeta(t) \int_A \expf{-i t^\top x} \d x \d t.
    \end{align*}
    Since $\zeta \in L^1(\Rp)$ and $A$ is bounded, the integrand in the previous equation is integrable, and the limit $T \rightarrow \infty$ can be replaced by the proper integral over $\Rp$. Further, using the absolute convergence property of $\zeta$, Fubini's Theorem allows us to change the order of integration and gives us
    \begin{align*}
        P(X \in A) 
        &= \left(2\pi\right)^{-p}\intRp{t}{\int_A \zeta(t) \expf{-i t^\top x} \d x} \\
        &= \int_A \left(2\pi\right)^{-p} \intRp{t}{\zeta(t) \expf{-i t^\top x} } \d x.
    \end{align*}
    By definition, this shows that the density of $X$ exists and is given by (\ref{eq-density-via-charfun}).
\end{proof}

If $X$ has a density function, the characteristic function of $X$ corresponds to the Fourier transform of its density. Taking this generalized view of Fourier transforms will allow us to study approximations of densities that are not necessarily densities and characteristic functions themselves. We will use a less commonly used definition of the Fourier transform, in which the sign of the exponent is reversed. For $f \in L^1(\Rp)$, the Fourier transform $\F[f]$ is the function given by
\begin{equation} \label{eq-fourier-density}
    \F[f](t) = \int_\Rp \expf{it^\top x}f(x)\d x \ \ \ \ \ \ \t{for all } t \in \Rp.
\end{equation}

In this context, we can generalize Theorem \ref{thm-char-inversion} to provide the necessary conditions under which the Fourier transform can be inverted. This corresponds to the extension of \cite[Corollary 2.4.3]{kolassa2006series} to $\Rp$.

\begin{corollary} \label{corr-fourier-inv}
    Suppose that $f \in L^1(\Rp)$ and $\zeta \in L^1(\Rp)$ are related by
    \begin{equation}\label{eq-fourier-trans}
        \zeta(t) = \F[f](t) \ \ \ \ \ \ \t{for all } t \in \Rp.
    \end{equation}
    Then, for any $x \in \Rp$, it holds that
    \begin{equation} \label{eq-fourier-inv}
        f(x) = \left(2\pi\right)^{-p}\int_\Rp \expf{-i t^\top x} \zeta(t) \d t.
    \end{equation}
\end{corollary}

\begin{proof}
    We decompose $f$ in positive and negative parts by $f(x) = f^+(x) - f^-(x)$ where $f^+(x) = f(x) \mathbb{1}_{f(x) \geq 0}$ and $f^-(x) = -f(x) \mathbb{1}_{f(x) < 0}$. Then, if $c^+ = \int_\Rp f^+(x) \d x$ and $c^- = \int_\Rp f^-(x) \d x$, the functions $f^+ / c^+$ and $f^- / c^-$ are both densities over $\Rp$ with resepctive characteristic function $\zeta^+$ and $\zeta^-$. We can replace these quantities in (\ref{eq-fourier-trans}) to have
    \begin{align*}
        \zeta(t) 
        &= \int_\Rp \expf{it^\top x}f(x)\d x \\
        &= c^+ \int_\Rp \expf{it^\top x} \frac{1}{c^+}f^+(x)\d x - c^- \int_\Rp \expf{it^\top x} \frac{1}{c^-}f^-(x)\d x\\
        &= c^+ \zeta^+(t) - c^-\zeta^-(t).
    \end{align*}
    By applying Theorem \ref{thm-char-inversion} to the positive and negative parts of $f$, we obtain that
    \begin{equation*}
        \frac{1}{c^\pm} f^\pm(x) = \left(2\pi\right)^{-p}\int_\Rp \expf{-i t^\top x} \zeta^\pm(t) \d t,
    \end{equation*}
    and hence
    \begin{align*}
        f(x) 
        &= f^+(x) - f^-(x) \\
        &= c^+ \left(2\pi\right)^{-p}\int_\Rp \expf{-i t^\top x} \zeta^+(t) \d t
         - c^- \left(2\pi\right)^{-p}\int_\Rp \expf{-i t^\top x} \zeta^-(t) \d t \\
        &= \left(2\pi\right)^{-p}\int_\Rp \expf{-i t^\top x} \left[ c^+\zeta^+(t) - c^-\zeta^-(t)\right] \d t\\
        &= \left(2\pi\right)^{-p}\int_\Rp \expf{-i t^\top x} \zeta(t) \d t. \qedhere
    \end{align*}
\end{proof}

This corollary lets us extend the notation introduced in (\ref{eq-fourier-density}) and define the inverse Fourier transform operator $\Finv$ as in (\ref{eq-fourier-inv}),
\begin{equation*}
    \Finv[\zeta](x) = \left(2\pi\right)^{-p}\int_\Rp \expf{-i t^\top x} \zeta(t) \d t.
\end{equation*}

In order to better understand the characteristic function, we need to be able to know in which functional space it lies. The following lemma relates $L^p$ integrability of the characteristic function to the existence of the density of a convolution of random variables. This is an extension of \cite[Lemma 2.4.4]{kolassa2006series} to $\Rp$.

\begin{lemma} \label{lem-char-integrable-convolution}
    The characteristic function $\xi$ of a random variable $X$ in $\Rp$ satisfies $\xi \in L^q(\Rp)$ for some $q > 1$ if and only if there exists a positive integer $l \in \N$ such that the density of a convolution of $l$ independent copies of $X$ exists and is bounded.
\end{lemma}
\begin{proof}
    This proof is an adaptation of the one-dimensional proof given in \cite[Lemma 2.4.4]{kolassa2006series}.

    The \textit{only if} direction is a direct consequence of Theorem \ref{thm-char-inversion}. Assuming that $\xi \in L^q(\Rp)$ we have that $\xi \in L^l(\Rp)$ for $l = \ceil{q}$ and hence,
    \begin{equation*}
        \intRp{t}{\abs{\xi(t)^l}} < \infty,
    \end{equation*}
    and so $\xi^l \in L^1(\Rp)$. Since $\xi^l$ is the characteristic function of a sum of $l$ independent copies of $X$, we can apply Theorem \ref{thm-char-inversion}, which gives us that the density of the convolution of $l$ copies of $X$ exists and is bounded.
    \newline
    We now prove the \textit{if} direction of the theorem. Assume that there exists a positive interger $j \in \N$ such that the density $f_j$ of a convolution of $j$ independent copies of $X$ exists and is bounded. Then, for any $r \in \R$,
    \begin{equation*}
        \int_{[-r, r]^p} \abs{\xi(t)}^{2j} \d t
        = \int_{[-r, r]^p} \abs{\xi(t)}^j\abs{\xi(t)}^j \d t
        = \int_{[-r, r]^p} \abs{\xi(t)}^j\abs{\xi(-t)}^j \d t,
    \end{equation*}
    where we use that $\abs{\xi(-t)} = \abs{\overline{\xi(t)}} = \abs{\xi(t)}$. Furthermore, by the definition of the characteristic function and using Fubini's theorem, we have that
    \begin{align*}
        \int_{[-r, r]^p} \abs{\xi(t)}^{2j} \d t
        &= \int_{[-r, r]^p} \left[\intRp{x}{f_j(x)\expf{it^\top x}}\right]\left[\intRp{y}{f_j(y)\expf{-it^\top y}}\right] \d t\\
        &= \intRp{x}{\intRp{y}{\int_{[-r, r]^p} f_j(x)f_j(y)\expf{it^\top (x - y)}\d t}}.
    \end{align*}
    Setting $z = y - x$ and using the identity $\sin x = (\expf{ix} - \expf{-ix}) / 2i$, we get
    \begin{align*}
        \int_{[-r, r]^p} \abs{\xi(t)}^{2j} \d t
        &= \intRp{x}{\intRp{z}{\int_{[-r, r]^p} f_j(x)f_j(x + z)\expf{-it^\top z}\d t}}\\
        &= \intRp{x}{\intRp{z}{f_j(x)f_j(x + z)\left[\prod_{k=1}^p \int_{[-r, r]}\expf{-it^\top z_k}\right]\d t}}\\
        &= \intRp{x}{\intRp{z}{f_j(x)f_j(x + z)\left[\prod_{k=1}^p \frac{\expf{-irz_k} - \expf{irz_k}}{-iz_k} \right]}}\\
        &= \intRp{x}{\intRp{z}{f_j(x)f_j(x + z)\left[\prod_{k=1}^p \frac{2 \sin(rz_k)}{z_k} \right]}}.
    \end{align*}
    Applying the change of variable $v = r z$ gives us
    \begin{align*}
        \int_{[-r, r]^p} \abs{\xi(t)}^{2j} \d t
        &= \intRp{x}{\intRp{v}{f_j(x)f_j(x + v/r)\left[\prod_{k=1}^p \frac{2 \sin(v_k)}{v_k/r} \right] r^{-p}}}\\
        &= \intRp{x}{\intRp{v}{f_j(x)f_j(x + v/r)\left[\prod_{k=1}^p \frac{2 \sin(v_k)}{v_k} \right]}}.
    \end{align*}
    Using the fact that $\sup_{x \in \R} \abs{\sin x / x} < 1$, we have that $\prod_{k=1}^p \frac{2 \sin(v_k)}{v_k} < 2^p$ and
    \begin{equation*}
        \int_{[-r, r]^p} \abs{\xi(t)}^{2j} \d t
        \leq \intRp{x}{f_j(x)\intRp{v}{f_j(x + v/r)}}
        = \intRp{x}{f_j(x)} = 1.
    \end{equation*}
    This shows that a finite upper bound on $\norm{\xi}_{2j}^{2j}$ that is independent of $r$ exists. This concludes the proof for $q = 2j$.
\end{proof}

In the following sections, we study approximations of the characteristic function in terms of its Taylor approximation. As one might expect, computing the Fourier and inverse Fourier transforms of such approximations involves computing the Fourier transforms of derivatives of the characteristic function. Before studying Fourier transforms of differential quantities, we introduce some notation for multivariate derivatives.

For $k \in \N$, we define $S_p(k)$ as the set of index vectors of length $k$ over $p$-dimensional vectors, that is,
\begin{equation*}
    S_p(k) = \left\{ (s_1, \ldots, s_k) : s_i \in [p] \right\},
\end{equation*}
where $[p] = \eset{1, \ldots, p}$. Let $f : \Rp \rightarrow \R$ be a $k$-times differentiable function, $s \in S_p(k)$ and $x_0 \in \Rp$, then the \textit{$s$-derivative} of $f$ in $x_0$ is given by
\begin{equation*}
    D^s f(x_0) = \frac{\d^k}{\d x_{s_1} \ldots \d x_{s_k}} f(x) \bigg|_{x=x_0}.
\end{equation*}

We now proceed to the following lemma, which gives a simple expression of the Fourier transform of derivatives of a function.

\begin{lemma} \label{lemma-fourier-derivative}
    Let $r \in \N$ and $f \in L^1(\Rp)$ such that all partial derivatives of $f$ of order up to $r$ exist, and for any $\tilde{s} \in S_p(r-1)$,
    \begin{equation} \label{eq-tails-to-zero}
        \lim_{\norm{x} \rightarrow \infty} \expf{it^\top x}D^{\tilde{s}}f(x) = 0.
    \end{equation}
    Then for any $s \in S_p(r)$, it holds that
    \begin{equation*}
        \F\left[D^s f \right](t) = (-i)^r t^s \F[f].
    \end{equation*}
\end{lemma}
\begin{proof}
    Let $\tilde{s} = (s_1, \ldots, s_{r-1})$, then by direct computation of the Fourier transform,
    \begin{align*}
        \F\left[D^s f \right](t) 
        &= (2\pi)^{-p}\intRp{x}{ \expf{it^\top x} D^s f(x) } \\
        &= (2\pi)^{-p}\int_{\R^{-1}} \int_\R 
            \expf{it^\top x} \ddx{x_{s_r}} D^{\tilde{s}} f(x) 
        \d x_{s_r} \d x_{\tilde{s}}\,.
    \end{align*}
    Integrating by part over the axis $x_{s_r}$ and using Assumption (\ref{eq-tails-to-zero}) gives,
    \begin{align*}
        \F\left[D^s f \right](t)  
        &= - (2\pi)^{-p}\intRp{x}{
            (it_{s_r})\expf{it^\top x} D^{\tilde{s}} f(x)
        } \\
        &= -it_{s_r} (2\pi)^{-p}\intRp{x}{
            \expf{it^\top x} D^{\tilde{s}} f(x)
        } \\
        &= -it_{s_r} \F\left[D^{\tilde{s}} f \right](t)
    \end{align*}
    Iterating the previous steps completes the proof.
\end{proof}

Next, we introduce the \textit{cumulant generating function}, a quantity related to the characteristic function that is easier to manipulate. For a random vector $X$ in $\Rp$, the cumulant generating function of $X$ is the function $K : \Rp \rightarrow \R$ given by
\begin{equation*}
    K(t) = \log \expec{}{\expf{t^\top X}}.
\end{equation*}

The derivatives of the cumulant generating function are called the \textit{cumulants}. Let $s \in S_p(k)$ be an index vector of length $k \in \N$, then if the involved derivatives exist, we define the $s$-cumulant of $X$ as
\begin{equation*}
    \kappa_s(X) = D^s K(0).
\end{equation*}
In the sequel, cumulants might depend on various quantities such as the sample size, the random variable of interest or parameters of a distribution, in which case we will use variations of the notation $\kappa_s$ to make clear which cumulants are being discussed.

Since the Normal distribution will often be used in the follwoing, the next example gives the cumulant generating function and cumulants of a multivarite Normal distribution.
\begin{example} \label{ex-cumulants-mvn}
    Let $X \sim N(\mu, \Sigma)$ with $\mu \in \Rp$ and $\Sigma \in \Sp$. The cumulant generating function $K(t; \mu, \Sigma)$ of $X$ is the quadratic function given by
    \begin{equation*}
        K(t; \mu, \Sigma) = t^\top\mu + t^\top\Sigma t.
    \end{equation*}
    From this, it is clear that all cumulants of $X$ exist. Furthermore, the first order cumulants are the components $\mu$, second order cumulants are the components of $\Sigma$, and the cumulants of higher order are 0.
\end{example}

We now state without a proof some simple properties of cumulants that will be useful in future proofs.
\begin{lemma} \label{lem-cumulants-props}
    Let $X_1, \ldots, X_n \simiid P$. Then the following holds for any $s \in S(k)$
    \begin{enumerate}[i.]
        \item {
        $\kappa_s(X_1 + \ldots + X_n) = n\kappa_s(X_1)$
        }
        \item {
            For all $c \in \R$, $\kappa_s(c X_1) = c^k\kappa_s(X_1)$
        }
        \item {
            For all $c \in \Rp$, $\kappa_s(X_1 + c) =
            \begin{cases}
                \kappa_s(X_1) + c_i &\text{if}\ s=(i),\\
                \kappa_s(X_1)& \text{otherwise},
            \end{cases}$
        }
    \end{enumerate}
    where we write $\kappa_s(Z)$ for the $s$-cumulant of the random variable $Z$.
\end{lemma}

Note that the cumulant generating function is closely related to the characteristic function since
\begin{equation*}
    K(t) 
    = \log \expec{}{\expf{t^\top X}} 
    = \log \expec{}{\expf{i (-i)t^\top X}}
    = \log \zeta(-it).
\end{equation*}
This equality allows us to define the cumulants $\kappa_s$ for $s \in S_p(k)$ in terms of the characteristic function
\begin{equation*}
    \kappa_s(X) = D^s K(0) 
    = \frac{\d^k}{\d x_{s_1} \ldots \d x_{s_k}} \log \zeta(-it) \bigg|_{t=0}
    = (-i)^{k} D^s \log \zeta(0),
\end{equation*}
and hence
\begin{equation*}
    D^s \log \zeta(0) = i^k \kappa_s(X).
\end{equation*}