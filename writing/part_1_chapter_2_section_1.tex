\section{Charlier differential Series}


\begin{definition}[Characteristic function]
    Let $X$ be a random vector in $\Rp$. The characteristic function of $X$ is the function $\zeta_X : \Rp \rightarrow \C$ given by
    \begin{equation*}
        \zeta_X(t) = \expec{}{\expf{it^\top X}}.
    \end{equation*}
\end{definition}

If $X$ has a density function, the characteristic function of $X$ is equal to the Fourier transform of its density. To facilitate future discussions, we introduce a special notation for the Fourier transform operator $\F$ that will applied both to distribution functions, then denoting the characteristic function, as
\begin{equation} \label{eq-fourier-distrib}
    \F[P](t) = \expec{}{\expf{it^\top X}},
\end{equation}
where $X \sim P$, but also applied to any function $f \in L^1(\Rp)$ with
\begin{equation} \label{eq-fourier-density}
    \F[f](t) = \int_\Rp \expf{it^\top x}f(x)\d x.
\end{equation}
The following theorem shows that under integrability condition, the Fourier transform can be inverted and the characteristic function be used to express the density of a distribution.

\begin{theorem} \label{thm-char-inversion}
    Let $X \sim P$ be a random vector in $\Rp$ with characteristic function $\zeta_X \in L^1(\Rp)$. Then, the density of $X$ exists and is given by
    \begin{equation} \label{eq-density-via-charfun}
        f_X(x) = \left(2\pi\right)^{-p} \intRp{t}{ \expf{-it^\top x}\zeta_X(t) }.
    \end{equation}
\end{theorem}

\begin{proof}
    Let $A \subset \Rp$ be a bounded rectangle $A = [a_1, b_1] \times \ldots \times [a_p, b_p]$ with $P(X \in \partial A) = 0$. By Theorem 3.10.4 in \cite{durrett_2019}, we have that
    \begin{equation*}
        P(X \in A) = \lim_{T \rightarrow \infty} \left(2\pi\right)^{-p}\int_{\left[-T, T\right]^p} \zeta_X(t) \prod_{k=1}^p \frac{\expf{-it_k a_k} - \expf{-it_k b_k}}{i t_k} \d t.
    \end{equation*}
    By rewriting various terms under the integral, one obtains
    \begin{align*}
        P(X \in A) 
        &= \lim_{T \rightarrow \infty} \left(2\pi\right)^{-p}\int_{\left[-T, T\right]^p} \zeta_X(t) \prod_{k=1}^p \frac{\expf{-it_k a_k} - \expf{-it_k b_k}}{i t_k} \d t \\
        &= \lim_{T \rightarrow \infty} \left(2\pi\right)^{-p}\int_{\left[-T, T\right]^p} \zeta_X(t) \prod_{k=1}^p \int_{a_k}^{b_k}\expf{-i t_k x_k} \d x_k \d t \\
        &= \lim_{T \rightarrow \infty} \left(2\pi\right)^{-p}\int_{\left[-T, T\right]^p} \zeta_X(t) \int_A \expf{-i t^\top x} \d x \d t.
    \end{align*}
    Since $\zeta_X \in L^1(\Rp)$ and since $A$ is bounded, the integrand in the previous equation is integrable, and the limit $T \rightarrow \infty$ can be replaced by the proper integral over $\Rp$. Furthermore, using the same absolute convergence property and by Fubini's Theorem the order of integration can be changed, yielding
    \begin{align*}
        P(X \in A) 
        &= \left(2\pi\right)^{-p}\intRp{t}{\int_A \zeta_X(t) \expf{-i t^\top x} \d x} \\
        &= \int_A \left(2\pi\right)^{-p} \intRp{t}{\zeta_X(t) \expf{-i t^\top x} } \d x.
    \end{align*}
    By definition, this shows that the density of $X$ exists and is given by (\ref{eq-density-via-charfun}).
\end{proof}

While being a useful result on its own, Theorem \ref{thm-char-inversion} can also be used to provide the necessary conditions to invert the Fourier transform of a function that is not a density.

\begin{corollary}
    Suppose that two uniformly integrable functions $f : \Rp \rightarrow \R$ and $\zeta : \Rp \rightarrow \C$ are related by
    \begin{equation}\label{eq-fourier-trans}
        \zeta(t) = \F[f](x).
    \end{equation}
    Then, it holds that
    \begin{equation} \label{eq-fourier-inv}
        f(x) = \left(2\pi\right)^{-p}\int_\Rp \expf{-i t^\top x} \zeta(t) \d t.
    \end{equation}
\end{corollary}

\begin{proof}
    We decompose $f$ in it's positive and negative part by $f(x) = f^+(x) - f^-(x)$ where $f^+(x) = f(x) \mathbb{1}_{f(x) \geq 0}$ and $f^-(x) = -f(x) \mathbb{1}_{f(x) < 0}$. Then, if $c^+ = \int_\Rp f^+(x) \d x$ and $c^- = \int_\Rp f^-(x) \d x$, the functions $f^+ / c^+$ and $f^- / c^-$ are both densities over $\Rp$ with characteristic functions $\zeta^+$ and $\zeta^-$. We can then replace these quantities in (\ref{eq-fourier-trans}) to have
    \begin{align*}
        \zeta(t) 
        &= \int_\Rp \expf{it^\top x}f(x)\d x \\
        &= c^+ \int_\Rp \expf{it^\top x} \frac{1}{c^+}f^+(x)\d x - c^- \int_\Rp \expf{it^\top x} \frac{1}{c^-}f^-(x)\d x\\
        &= c^+ \zeta^+(t) - c^-\zeta^-(t).
    \end{align*}
    By applying Theorem \ref{thm-char-inversion} to the positive and negative parts of $f$, we obtain that
    \begin{equation*}
        \frac{1}{c^\pm} f^\pm(x) = \left(2\pi\right)^{-p}\int_\Rp \expf{-i t^\top x} \zeta^\pm(t) \d t,
    \end{equation*}
    and hence
    \begin{align*}
        f(x) 
        &= f^+(x) - f^-(x) \\
        &= c^+ \left(2\pi\right)^{-p}\int_\Rp \expf{-i t^\top x} \zeta^+(t) \d t
         - c^- \left(2\pi\right)^{-p}\int_\Rp \expf{-i t^\top x} \zeta^-(t) \d t \\
        &= \left(2\pi\right)^{-p}\int_\Rp \expf{-i t^\top x} \left[ c^+\zeta^+(t) - c^-\zeta^-(t)\right] \d t\\
        &= \left(2\pi\right)^{-p}\int_\Rp \expf{-i t^\top x} \zeta(t) \d t.
    \end{align*}
\end{proof}

This theorem lets us extend the notation introduced in Equation (\ref{eq-fourier-density}) and define the inverse Fourier transform operator $\Finv$ as in Equation (\ref{eq-fourier-inv}),
\begin{equation*}
    \Finv[\zeta](x) = \left(2\pi\right)^{-p}\int_\Rp \expf{-i t^\top x} \zeta(t) \d t.
\end{equation*}


\begin{lemma} \label{lemma-fourier-derivative}
    Let $r \in \N$ and $f : \R \rightarrow \Rp$ such that all partial derivatives of $f$ of order up to $r$ exist and for any $\tilde{s} \in S(r-1)$
    \begin{equation} \label{eq-tails-to-zero}
        \lim_{\norm{x} \rightarrow \infty} \expf{it^\top x}D^{\tilde{s}}f(x) = 0.
    \end{equation}
    Then for any $s \in S(r)$, it holds that
    \begin{equation*}
        \F\left[D^s f \right](t) = (-i)^r t^s \F[f].
    \end{equation*}
\end{lemma}
\begin{proof}
    Let $\tilde{s} = (s_1, \ldots, s_{r-1})$, then, by direct computation of the Fourier transform,
    \begin{align*}
        \F\left[D^s f \right](t) 
        &= (2\pi)^{-p}\intRp{x}{ \expf{it^\top x} D^s f(x) } \\
        &= (2\pi)^{-p}\int_{\R^{-1}} \int_\R 
            \expf{it^\top x} \ddx{x_{s_r}} D^{\tilde{s}} f(x) 
        \d x_{s_r} \d x_{\tilde{s}}.
    \end{align*}
    Integrating by part over the axis $x_{s_r}$ and using Assumption (\ref{eq-tails-to-zero}) gives
    \begin{align*}
        \F\left[D^s f \right](t)  
        &= - (2\pi)^{-p}\intRp{x}{
            (it_{s_r})\expf{it^\top x} D^{\tilde{s}} f(x)
        } \\
        &= -it_{s_r} (2\pi)^{-p}\intRp{x}{
            \expf{it^\top x} D^{\tilde{s}} f(x)
        } \\
        &= -it_{s_r} \F\left[D^{\tilde{s}} f \right](t)
    \end{align*}
    Iterating the previous steps completes the proof.
\end{proof}


\begin{definition}
    The cumulant generating function $K_X : \Rp \rightarrow \R$ of a random vector $X \in \Rp$ is defined as 
    \begin{equation*}
        K_X(t) = \log \expec{X}{\expf{t^\top X}}.
    \end{equation*}
    Let $s \in S_p(k)$ be an index vector of length $k$, then if the involved derivatives exist, we define the $s$-cumulant of $X$ as
    \begin{equation*}
        \kappa_s = D^s K(0).
    \end{equation*}
\end{definition}

\begin{remark}
    One can see that the cumulant generating function is closely related to the characteristic function since
    \begin{equation*}
        K_X(t) 
        = \log \expec{X}{\expf{t^\top X}} 
        = \log \expec{X}{\expf{i (-i)t^\top X}}
        = \logf{\zeta_X(-it)}.
    \end{equation*}

    This equality also allows us to define the cumulants $\kappa_s$ for $s \in S_p(k)$ in terms of the characteristic function
    \begin{equation*}
        \kappa_s = D^s K(0) 
        = \frac{\d^k}{\d x_{s_1} \ldots \d x_{s_k}} \log \zeta(-it) \bigg|_{t=0}
        = (-i)^{k} D^s \log \zeta(0).
    \end{equation*}
\end{remark}

We now present a heuristic development of the idea behind the Edgeworth expansion. Consider two distributions $P$ and $Q$ over $\Rp$ with densities $f$ and $q$, characteristic functions $\zeta$ and $\xi$, and cumulants $\kappa_s$ and $\gamma_s$ for $s \in S(k)$, $k \in \N$. We wish to utilize the cumulants of both distribution to construct an approximation of $P$.

By formal expansion of the difference between the cumulant generating functions of $P$ and $Q$ around 0, we obtain for any $t \in \Rp$
\begin{align*}
    \log \frac{\zeta(t)}{\xi(t)}
    = \log \zeta(t) - \log \xi(t) 
    &= \sum_{r=0}^\infty \sum_{s \in S(r)} (\kappa_s - \gamma_s)\frac{(-i)^r t^s}{r!}\\
    &= \sum_{r=1}^\infty \sum_{s \in S(r)} (\kappa_s - \gamma_s)\frac{(-i)^r t^s}{r!},
\end{align*}
where the last equality holds from $\zeta(0) = 1 = \xi(0)$. Exponentiating on both side and isolating $\zeta(t)$, we find that
\begin{equation*}
    \zeta(t) = \xi(t)\expfc{\sum_{r=1}^\infty \sum_{s \in S(r)} (\kappa_s - \gamma_s)\frac{(-i)^r t^s}{r!}}.
\end{equation*}

Let $\alpha^s = \kappa_s - \gamma_s$, we can then continue by taking a formal expansion of the exponential function to find
\begin{align*}
    \zeta(t)
    &= \xi(t)\expfc{\sum_{r=1}^\infty \sum_{s \in S(r)} \alpha^s\frac{(-i)^r t^s}{r!}}\\
    &= \xi(t)\sum_{j=0}^\infty \frac{1}{j!} \left\{\sum_{r=1}^\infty \sum_{s \in S(r)} \alpha^s\frac{(-i)^r t^s}{r!}\right\}^j \\
    &=
    \sum_{j=0}^\infty \frac{1}{j!} 
    \sum_{\substack{r_1 = 1\\ \ldots \\r_j = 1}}^\infty
    \sum_{\substack{s_1 \in S(r_1)\\ \ldots \\s_j \in S(r_j)}}
    \alpha^{s_1}\ldots\alpha^{s_j}
    \frac{
        \xi(t) (-i)^{r_1 + \ldots + r_j}
        t^{s_1} \ldots t^{s_j}
    }{
        r_1! \ldots r_j!
    }.
\end{align*}

By Lemma \ref{lemma-fourier-derivative}, one recognizes the Fourier transform of derivatives of the density $q$ of $Q$
\begin{equation*}
    \xi(t) (-i)^{r_1 + \ldots + r_j}
        t^{s_1} \ldots t^{s_j} = \F \left[ D^{s_1} \ldots D^{s_j} q \right].
\end{equation*}
The density of $P$ can thus be retrieved by Fourier inversion, giving
\begin{equation*}
    f(x) 
    = \sum_{j=0}^\infty \frac{1}{j!} 
        \sum_{\substack{r_1 = 1\\ \ldots \\r_j = 1}}^\infty
        \sum_{\substack{s_1 \in S(r_1)\\ \ldots \\s_j \in S(r_j)}}
        \alpha^{s_1}\ldots\alpha^{s_j}
        \frac{
            D^{s_1} \ldots D^{s_j} q(x)
        }{
            r_1! \ldots r_j!
        }.
\end{equation*}



\section{Properties of characteristic functions}

Throughout this chapter, we consider a family of multivariate continuous distributions $\{ P_\theta \}$ indexed by a parameter $\theta \in \Theta$.



\begin{lemma} \label{lemma-series}
    For any $u, v \in \C$ and any $l \in \mathbb{N}$ the following inequality holds
    \begin{equation}
        \abs{\exp{u} - \sum_{k=0}^l \frac{v^k}{k!}} \leq \max \left\{ \exp{|u|}, \exp{|v|} \right\}\left(\abs{u - v} + \abs{\frac{v^{l+1}}{(l+1)!}}\right).
    \end{equation}
\end{lemma}

\begin{proof}
    By the triangle inequality,
    \begin{equation*}
        \abs{\exp{u} - \sum_{k=0}^l \frac{v^k}{k!}} \leq \abs{\exp{u} - \exp{v}} + \abs{\sum_{k=l+1}^\infty \frac{v^k}{k!}}.
    \end{equation*}
    Starting with the second term on the right hand side, we have
    \begin{align*}
        \abs{\sum_{k=l+1}^\infty \frac{v^k}{k!}} 
        & \leq \sum_{k=l+1}^\infty \abs{\frac{v^k}{k!}} 
        = \abs{\frac{v^{l+1}}{(l+1)!}} \sum_{k=0}^\infty \abs{v^k} \frac{(l+1)!}{(k + l + 1)!} \\
        & \leq \abs{\frac{v^{l+1}}{(l+1)!}} \sum_{k=0}^\infty \abs{\frac{v^k}{k!}}
        = \abs{\frac{v^{l+1}}{(l+1)!}} \exp v\\
        &\leq \abs{\frac{v^{l+1}}{(l+1)!}} \max \left\{ \exp{|u|}, \exp{|v|} \right\}.
    \end{align*}
    Furthermore, by Taylor's theorem there exists a point $z \in \C$ lying on the straight line between $u$ and $v$ such that
    \begin{equation*}
        \exp u - \exp v = \abs{u - v} \exp z.
    \end{equation*}
    Taking absolute values and by convexity of the exponential function, we find the following bound
    \begin{equation*}
        \abs{\exp u - \exp v} = \abs{u - v} \exp |z| \leq \abs{u - v} \max \left\{ \exp{|u|}, \exp{|v|} \right\}.
    \end{equation*}
    Combining the two bounds found previously completes the proof of the lemma.
\end{proof}


\begin{theorem}
    Let $\zeta_X$ be the characteristic function of a random vector $X \in \Rp$ and let $n \in \N$. Assume that all cumulants of $X$ of order up to $k \in \N$ exist and that the second cumulants of $X$ satisfy $\kappa^{(i, j)} = \delta_{ij}$ for all $i, j = 1, \ldots, p$. Let
    \begin{equation*}
        \xi(t) = \expf{-\frac{1}{2}\norm{t}_2^2}\sum_{l=0}^{k-2} \frac{1}{l!} \left[ 1 + \sum_{m=3}^k \sum_{s \in S(m)} \frac{i^m\kappa_s t_{s_1}\ldots t_{s_m}}{n^{m/2-1}m!} \right]^l.
    \end{equation*}
    Then for every $\epsilon > 0$ there exists a $\delta > 0$ and a constant $C_p$ dependent on the dimension of $X$ such that 
    \begin{equation}\label{eq-edge-bound}
        \abs{\expf{n\left[\log \zeta_X(t n^{-1/2})\right]} - \xi(t)} \leq \expf{-\frac{1}{4}\norm{t}_2^2}\left[ \frac{\epsilon\norm{t}_2^k}{n^{k/2-1}} + \frac{C_p^{k-1}\norm{t}_2^{3(k-1)}}{(k-1)!n^{k/2-1/2}} \right]
    \end{equation}
    holds for all $t \in \Rp$ with $\norm{t}_2 < \delta\sqrt{n}$.
\end{theorem}
\begin{proof}
    The idea of this proof is to rewrite the left hand side of the Equation (\ref{eq-edge-bound}) to be able to use Lemma \ref{lemma-series} and find suitable upper bounds on the remaining quantities. To that end, we define $u(t) = n u^*(t n^{-1/2})$ and $v(t) = nv^*(t n^{-1/2})$ where $u^*(t) = \log \zeta_X(t) + \frac{1}{2}\norm{t}_2^2$ and 
    \begin{equation*}
        v^*(t) = \sum_{m=3}^k \sum_{s \in S(m)} \frac{i^m\kappa_s t_{s_1}\ldots t_{s_m}}{m!}.
    \end{equation*}
    We can then rewrite
    \begin{equation*}
        \expf{n\left[\log \zeta_X(t n^{-1/2})\right]} = \expf{-\frac{1}{2}\norm{t}_2^2}\exp u(t)
    \end{equation*}
    and
    \begin{equation*}
        \xi(t) = \expf{-\frac{1}{2}\norm{t}_2^2}\sum_{l=0}^{k-2} \frac{v(t)^l}{l!}.
    \end{equation*}
    By Lemma \ref{lemma-series}, we can bound the left hand side of Equation (\ref{eq-edge-bound}) and have
    \begin{align*}
        &\abs{\expf{n\left[\log \zeta_X(t n^{-1/2})\right]} - \xi(t)} \\
        &= \expf{-\frac{1}{2}\norm{t}_2^2}\abs{\exp u(t) - \sum_{l=0}^{k-2} \frac{v(t)^l}{l!}}\\
        &\leq \expf{-\frac{1}{2}\norm{t}_2^2} \max \left\{ \exp{|u(t)|}, \exp{|v(t)|} \right\}\left(\abs{u(t) - v(t)} + \frac{|v(t)|^{k-1}}{(k-1)!}\right)
    \end{align*}
    We now continue to find suitable bounds on the resulting quantities. First note that both $u$ and $v$ have continuous derivatives in 0 of order up to $k$. Starting with $u^*$, let $s \in S(m)$ for $3 \leq m < k$, then
    \begin{align*}
        \ddx{t_{s_1}}\ldots\ddx{t_{s_m}} u^*(t) \bigg|_{t=0}
        &= \ddx{t_{s_1}}\ldots\ddx{t_{s_m}} \log \zeta_X(t) + \frac{1}{2}\norm{t}_2^2\bigg|_{t=0}\\
        &=  \ddx{t_{s_1}}\ldots\ddx{t_{s_m}} \log \zeta_X(t)\bigg|_{t=0}
            +
            \ddx{t_{s_1}}\ldots\ddx{t_{s_m}} \frac{1}{2}\norm{t}_2^2\bigg|_{t=0}
        \\
        &= (-i)^{-m}\kappa_s = i^m \kappa_s ,
    \end{align*}
    where the derivative of the 2-norm of $t$ is 0 because $|s| = m \geq 3$. Furthermore, we can compute the same derivatives of $v^*$,
    \begin{align*}
        \ddx{t_{s_1}}\ldots\ddx{t_{s_m}} v^*(t) \bigg|_{t=0}
        &= \ddx{t_{s_1}}\ldots\ddx{t_{s_m}} \sum_{m'=3}^k \sum_{s' \in S(m')} \frac{i^{m'}\kappa^{s'}t_{s'_1}\ldots t_{s'_{m'}}}{m'!} \bigg|_{t=0} \\
        &= \sum_{m'=3}^k \sum_{s' \in S(m')} \frac{i^{m'}\kappa^{s'}}{m'!} \ddx{t_{s_1}}\ldots\ddx{t_{s_m}}t_{s'_1}\ldots t_{s'_{m'}} \bigg|_{t=0}.
    \end{align*}
    The term $\ddx{t_{s_1}}\ldots\ddx{t_{s_m}}t_{s'_1}\ldots t_{s'_{m'}} |_{t=0} = 1$ if and only if $s'$ is a permutation of $s$, otherwise 0. Hence
    \begin{equation*}
        \ddx{t_{s_1}}\ldots\ddx{t_{s_m}} v^*(t) \bigg|_{t=0} 
        = \sum_{s' \in S(m)} \frac{i^m\kappa^{s'}}{m!}
        = i^m\kappa_s,
    \end{equation*}
    where the last equality holds since $\kappa_s = \kappa^{s'}$ for all permutation $s'$ of $s$. 
    This shows that all derivatives of order up to $k$ of $u^* - v^*$ (and hence also $u - v$) exist in 0 and are all equal to 0. Therefore, there exists a $\delta > 0$ such that for all $t \in \Rp$ with $\norm{t}_2 \leq \delta$
    \begin{equation*}
        \abs{u^*(t) - v^*(t)} \leq \epsilon\norm{t}_2^k
    \end{equation*}
    and if $\norm{t} \leq \delta \sqrt{n}$, this bound yields
    \begin{equation*}
        \abs{u(t) - v(t)} = n\abs{u^*(t n^{-1/2}) - v(t n^{-1/2})} \leq n\epsilon\norm{n^{-1/2}t}_2^k = n^{1-k/2}\norm{t}_2^k\epsilon.
    \end{equation*}
    Furthermore, choose $\delta$ small enough such that $\abs{u^*(t)} < \norm{t}_2^2/4$ for $\norm{t}_2 \leq \delta$. Then for $\norm{t}_2 \leq \delta\sqrt{n}$ we have
    \begin{equation*}
        \abs{u(t)} = n\abs{u^*(t n^{-1/2})} \leq n \norm{t n^{-1/2}}_2^2/4 = \norm{t}_2^2/4.
    \end{equation*}
    We observe that all derivatives of $v^*$ in 0 of first and second order are equal to 0 and that the third derivatives of $v^*$ are bounded. By Taylor's theorem, this allows us to find the following bound for all $\norm{t}_2 \leq \delta \sqrt{n}$
    \begin{equation*}
        |v(t)| = |nv^*(t n^{-1/2})| < C_p n \norm{t n^{-1/2}}_2^3 = C_p \norm{t}_2^3 n^{-1/2}
    \end{equation*}
    where
    \begin{equation*}
        C_p = \sup_{\substack{\norm{t}_2 \leq \delta, \\ i,j,k \in \{1,\ldots,p\}}} p^3\abs{ \frac{\text{d}^3v^*(t)}{\text{d}t_i\text{d}t_j\text{d}t_k}}.
    \end{equation*}
    Hence, for a $\delta$ small enough and $t \leq \delta\sqrt{n}$ we have
    \begin{align*}
        &\abs{\expf{n\left[\log \zeta_X(t n^{-1/2})\right]} - \xi(t)} \\
        &\leq \expf{-\frac{1}{2}\norm{t}_2^2} \max \left\{ \exp{|u(t)|}, \exp{|v(t)|} \right\}\left(\abs{u(t) - v(t)} + \frac{|v(t)|^{k-1}}{(k-1)!}\right)\\
        &\leq \expf{-\frac{1}{2}\norm{t}_2^2}
            \max \left\{ 
                \expf{\frac{1}{4}\norm{t}_2^2},
                \expf{C \norm{t}_2^3 n^{-1/2}}
            \right\}\\
        & \times \left(
                \frac{\epsilon\norm{t}_2^k}{n^{k/2-1}}
                +
                \frac{C_p^{k-1} \norm{t}_2^{3(k-1)} }{n^{k/2-1/2} k!}
              \right)\\
        & \leq \expf{-\frac{1}{4}\norm{t}_2^2}\left[ \frac{\epsilon\norm{t}_2^k}{n^{k/2-1}} + \frac{C_p^{k-1}\norm{t}_2^{3(k-1)}}{(k-1)!n^{k/2-1/2}} \right].
    \end{align*}
\end{proof}

Note that this proof doesn't use the fact that $\zeta_X$ is a characteristic function. Indeed, this theorem can be proven in a more general setting without any changes to the statement of the theorem or the proof itself.